

\section{CompSoc Discord Server}
\subsection{Purpose and Membership}
\begin{enumerate}
    \item The CompSoc Discord server shall serve as an online platform for CompSoc members to communicate, collaborate, and socialize.
    \item All CompSoc members shall be eligible to join the Discord server, subject to agreeing to the server's code of conduct and rules.
\end{enumerate}

\subsection{Discord Moderation Team}
\begin{enumerate}
    \item The CompSoc Discord server shall have a dedicated moderation team consisting of both current CompSoc committee members and non-committee members (hereinafter referred to as "community moderators").
    \item Community moderators are added as provision intended to provide a system of checks and balances, ensuring that the Discord server remains a safe and welcoming space for all members, even in the event of a malicious committee.
    \item The moderation team shall be responsible for enforcing the server's code of conduct, maintaining a safe and inclusive environment, and ensuring the smooth operation of the server.
    \item The moderation team shall have two levels of access:
    \begin{itemize}
        \item Level 1 moderators shall have basic moderation powers, such as removing spam and issuing warnings.
        \item Level 2 moderators shall have additional powers, including server administration and the ability to ban or timeout users.
    \end{itemize}
    \item The CompSoc committee shall appoint a minimum of 3 committee members to the moderation team.
    \item The moderation team shall also include a minimum of 3 community moderators. 
\end{enumerate}

\subsection{Moderator Appointment and Removal}
\begin{enumerate}
    \item The appointment of new moderators shall require the approval of both the CompSoc committee and the current moderation team, with a majority vote in favor from each group.
    \item The quantity of moderators from both the CompSoc committee and the community should not differ by more than one.
    \item Moderators may be removed from the team if they are inactive for an extended period (e.g., no posts or moderation actions within a month) or if they misuse their moderation powers.
    \item The removal of moderators shall require the approval of both the CompSoc committee and the current moderation team (excluding the moderator being removed), with a two-thirds majority vote in favor from each group.
    \item Moderators who are removed may be reinstated at a later time if they wish to return and are approved by both the committee and moderation team.
\end{enumerate}

\subsection{Moderation Guidelines and Consequences}
\begin{enumerate}
    \item The moderation team shall establish and maintain clear guidelines for server conduct and moderation practices, which shall be publicly available to all server members.
    \item Moderators who misuse their powers or violate the moderation guidelines shall be subject to consequences, such as temporary suspension or permanent removal from the moderation team.
    \item The moderation team shall implement a strike system for moderator misconduct, with three strikes resulting in permanent removal from the team.
    \item Community moderators should refrain from interfering with committee matters unless deemed necessary.
\end{enumerate}

\subsection{Protected Channels}
\begin{enumerate}
    \item The CompSoc Discord server shall have designated protected channels for discussions related to sensitive topics or marginalized communities.
    \item Access to protected channels shall be granted through a separate vetting process determined by the moderation team and the relevant community members.
    \item Protected channels shall have their own dedicated moderators, who may or may not be part of the main CompSoc Discord moderation team.
    \item The CompSoc committee shall not have automatic access to protected channels, unless they are vetted and approved by the relevant community.
\end{enumerate}