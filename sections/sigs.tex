\section{Special Interest Groups (SIGs)}

\begin{enumerate}

\item Any member of CompSoc may, and is encouraged to, start a special interest group pertaining to any common interest the members may have. What follows are guidelines on the rights and rules regarding SIGs.

\item When forming a new SIG a request should be handed in to the committee outlining the main goals and purpose of the SIG\@.

\item The current special interest groups include:
  \begin{itemize}
  \item SIGWeb: Web Development
  \item HackSIG\@: Hackathons
  \item SIGINT\@: Cyber Security Group
  \item SIGnet: Computer Networking
  \end{itemize}

\item The following points will be considered by the committee when processing the application of a new SIG\@:
  \begin{enumerate}
  \item The SIG should have a clearly defined goal, purpose or other reason for existence deemed appropriate by the committee.
  \item The SIG should have at least 4 members. In extraordinary circumstances this can be overruled by the vote of 2/3 of the committee.
  \item The SIG should have current CompSoc-member as a leader. The leader shall be responsible for running the group and reporting back to the Vice President and the CompSoc committee. It is encouraged that the selection of a leader should be done democratically within the group.
  \end{enumerate}

\item New SIGs must be approved by two thirds of the elected committee.

\item Once an SIG is accepted:
  \begin{enumerate}
  \item The leader of any accepted SIG will automatically have a seat and a vote on the CompSoc committee.
  \item It is entirely up to the SIG how it wants to handle signing up members, arranging meetings or other matters pertaining to the general administration of the SIG\@.
  \item The SIG has the right to receive support from the committee within reason, including but not limited to financial support, creation of advertising materials and free promotion on the CompSoc website, newsletter and Facebook-group. Requests for such support should be given in writing to the committee.
  \item The leader of a SIG should continuously keep the committee informed of the activities of the SIG\@.
  \end{enumerate}

\item At the discretion of the executive committee, once a particular SIG has ran enough events, the SIG may be included in the constitution.

\item If the committee feels a SIG's activities has become incompatible with the interests and aims of the society, the elected committee may, by a two third majority, decide to dis-associate CompSoc from the SIG\@.

  \subsection{Leadership of SIGs}

\item The choice of a leader should be entirely up the SIG\@. If a major dispute about leadership of an SIG is brought to the attention of the committee, a member of the committee will temporarily act as a leader for the SIG and attempt to find a solution. If no agreement is made within a reasonable amount of time the committee may choose to appoint a leader, or decide to dis-credit the SIG\@.

\item When an SIG gets a new leader the change in leadership should be reported to the committee by the old leader.

  \subsection{Special Cases}

\item Note that the above are guidelines which should be followed in the general case. In some instances it will be impractical for an SIG to follow the above rules. This may include external groups or clubs with a university division that wishes to associate itself with CompSoc or on-campus groups associated with larger tech-companies. In these cases it will be entirely at the committee's discretion how the relationship between the SIG and CompSoc will work. The SIG will still require a two third majority of the committee to be approved.

\end{enumerate}
