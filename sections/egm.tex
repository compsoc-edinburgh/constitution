\section{Emergency General Meeting (EGM)}

\begin{enumerate}
\item An EGM can be called in the following ways:
  \begin{enumerate}
  \item The resignation of any elected committee-member will trigger an EGM at the earliest opportunity unless an AGM is held within 4 weeks.
  \item A quorate committee may vote, by simple majority, to hold an EGM on constitutional amendments.
  \item A quorate committee may vote, by two thirds majority, to hold an EGM to re-elect a committee member. This member would be entitled to run again for the same position and if re-elected could not be removed from the position by another EGM until an AGM is held.
  \item Any member of the society may bring about an EGM to amend the constitution or replace an elected member/members of the committee by collecting signatures of at least a third of the society, with a minimum of 30 members.
  \end{enumerate}

\item The committee shall decide on a date for an EGM which must be within 4 weeks of it being called for. If an EGM is called for outside of term-time the committee must decide on a date within 4 weeks of the start of term.

\item All members must receive at least 14 days written notification of an EGM\@. An e-mail suffices as a written notification.

\item An EGM can be called with the purpose of either amending the constitution or re-electing a committee member.

\item The quorum of a general meeting shall be the same as that for an AGM\@.

\item Any newly elected Office Bearers will be communicated to the Societies Team after the election has taken place.

\item The administrative running of an EGM shall be identical to that of an AGM\@.

\end{enumerate}
