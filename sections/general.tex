\section{General}

The name of the Society shall be \enquote{CompSoc} or \enquote{The University of Edinburgh Technology Society}.

\begin{enumerate}

\item The aims of the Society shall be as follows:
  \begin{itemize}
  \item To provide a forum for members to discuss issues relating to computer science and computing in general;
  \item To facilitate social interaction amongst people with a common interest in computer-related issues;
  \item To provide members with assistance with computer-related problems, including support for study;
  \item To promote liaison between the local academic and business communities, with regard to fostering future employment opportunities;
  \item Share experience and knowledge of work in Informatics by:
    \begin{itemize}
    \item Holding special interest group meetings to provide members with a platform for discussion and an opportunity to learn new skills.
    \item Holding informal seminars with workers and researchers in Informatics from industry or other academic institutions.
    \item Allowing members to have access to various Informatics-related programming tools in order to gain experience in using them to solve related problems.
    \end{itemize}
  \end{itemize}

\item The benefits of the Society can be:
  \begin{itemize}
  \item Free Pizza
  \item Free freebies
  \item Allow likeminded individuals the chance to socialise and network
  \item Allow the indulgence of all geekery
  \item Organised events to benefit those interested in technology
  \end{itemize}

\item The society shall abide by any applicable laws, bylaws and guidelines of the Edinburgh University Students' Association in relation to recognised societies.

\item Membership shall be open to all matriculated students of Edinburgh University.

\item Non-students may be members of CompSoc, vote in general meetings and stand for committee-positions other than President, Secretary and Treasurer.

\item Membership shall be at least 75\% matriculated students of Edinburgh University.

\item Full membership lists should be filed with the Activities Office at least once per semester to ensure membership databases are up to date.

\item If any dispute of membership should arise, the list registered through EUSA's database will be used as the official list.

\item All members who are not matriculated students of a University shall pay at least twice the annual subscription.

\item The society's bank account must have the President, Secretary, and Treasurer added as signatories.

\item The society has taken and will continue to take all necessary steps to ensure that our meetings, events and socials are accessible to all, irrespective of any disability.

\item The society has ensured and will continue to ensure that it complies with any relevant data protection legislation.

\item EUSA considers the ruling society constitution to be that which is displayed on the Society Profile.

\item Re-registration of the society must be submitted prior to re-registration deadline set by EUSA\@.

\item The financial year shall run from 1st April to 31st March.

\item The society shall be non-profit making. The Office Bearers and members may only receive payment, direct or indirect, as reimbursement for legitimate expenses.

\item The society believes that discrimination or harassment, direct or indirect, based on a person's gender, age (except where it relates to licensing laws), race, skin colour, nationality, religious belief, socioeconomic background, disability, HIV status, sexual orientation, gender reassignment, family situation, domestic responsibilities or any other irrelevant distinction, is detrimental to the society, the University and wider society, and will not be tolerated. As such the society operates a safe space policy.

\item The society may not register to obtain any legal status, including a limited company or a charitable status.

\item The Students' Association has a Zero Tolerance policy for sexual harassment and violence. This means that any case of sexual harassment or violence will be escalated to the final disciplinary stage (removal). Appeals will go through the formal complaints process of the Students' Association.

\item The Students' Association understands harassment to include all forms, whether expressed orally, in writing, or on any cyber or digital platform.

\item The Committee may impose sanctions for misconduct on members, based on verifiable evidence collected, taking into account the seriousness of the misconduct with reference to the membership responsibilities and standard code of conduct, any previous warnings, and any mitigating circumstances. When appropriate the following sanctions may be applied:
  \begin{enumerate}
  \item Informal Warning;
  \item First Written Warnings will be issued for a minor offence or complaint;
  \item Final Written Warnings will be issued for: a further offence or complaint or if the conduct of the member failed to improve following a previous warning, or if the actions are serious enough to warrant a Final Written Warning.
  \item Removal from membership will occur if there is a further serious event of misconduct, or if the conduct of the member failed to improve following a previous written warning, or if the member committed an act of gross misconduct. Examples of gross misconduct include but are not limited to discrimination, sexual harrassment, and any form of violence. The committee may proceed to removal of membership in cases of gross misconduct without having to go through warnings.
  \item EUSA's Societies department will be notified upon removal of a member and provided with a copy of the evidence pertaining to the sanctions process and removal. All persons removed from membership may appeal to the Societies' department against such removal.
  \end{enumerate}

\item The society shall remain \enquote{Most Excellent}.

\end{enumerate}
