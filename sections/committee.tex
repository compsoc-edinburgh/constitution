\section{The Committee}

\begin{enumerate}

\item All office-bearers shall be subject to election annually.

\item The General Committee
  \begin{itemize}
  \item The business of the Society shall be managed by a committee of Office Bearers.
  \item Any member of the Society shall be entitled to sit on the committee.
  \item The Office Bearers must be members of the Society.
  \item All Office Bearers will complete annual online training as outlined by the Activities Team.
  \end{itemize}

\item The Executive Committee
  \begin{itemize}
  \item The President, Secretary and Treasurer of the society shall be matriculated students of The University of Edinburgh.
  \item The committee must consist of a President, Secretary, and Treasurer (the Executive Officers) as a minimum.
  \item These Office Bearers will be elected at the AGM\@.
  \item No person may be elected to more than one of these posts simultaneously.
  \item Should any of the Executive Committee resign a replacement or acting replacement will be voted in at an EGM\@.
  \end{itemize}

\item Elected committee positions are as follows:
  \begin{enumerate}
  \item The President shall be ultimately responsible for the conduct of the society.
    \begin{itemize}
    \item The President will chair the General Committee and EGMs.
    \item The President is responsible to the AGM and the General Committee and is ultimately responsible for the conduct of the Society.
    \item It is the President's responsibility to ensure the Society has submitted their annual report, risk assessment and reregistration forms to the Activities Office.
    \end{itemize}
  \item The Secretary shall be responsible to the President for the administration of the society.
    \begin{itemize}
    \item The Secretary shall be responsible for the administration of the society.
    \item The Secretary shall also be responsible for any correspondence within or on behalf of the Society and prepare the agendas and the minutes of every committee meeting, AGM and EGM\@.
    \end{itemize}
  \item The Treasurer shall be responsible to the President for the finances of the society.
    \begin{itemize}
    \item The Treasurer shall be accountable to the committee and members for the finances of the society.
    \item The Treasurer shall keep and prepare Accounts of the Society and provide a provisional budget, as exhaustive as possible, to be presented at the AGM\@.
    \item It is the Treasurer's responsibility to ensure the annual financial report is complete and submitted along with the society's annual report.
    \end{itemize}
  \item Other elected positions on the committee may be Vice President, Hack Secretary, Social Secretary, Technical Secretary, Second Year Representative, Third Year Representative, Fourth Year Representative and Graphic Designer.
  \end{enumerate}

\item The year representatives must be students which will be enrolled in that year the semester following the AGM\@. At the time of the AGM the Second Year Representative should be a first year, etc.

\item The First Year (Freshers') Representative is a special case and will be elected at the first STMU (Or EGM) in the first semester before the end of October. If the position is not filled the committee may select a willing first year or hold subsequent re-elections.

\item The following are guidelines as to what role the person in these positions shall have in the society:
  \begin{enumerate}
  \item The Vice President shall oversee and work with the SIGs as well as being responsible for the STMUs (Finding speakers and handling venues).
  \item The committee may find volunteers responsible for promotional materials being made for events. The Secretary shall be ultimately responsible for news and updates being posted to the website and maintaining any social networks CompSoc may have a presence on, although this may be delegated to others if the Secretary feels this would be beneficial.
  \item The Secretary is also ultimately responsible for the creation of any society apparel such as hoodies, although this may be delegated to others if the Secretary feels this would be beneficial. The Social Secretary shall be responsible for booking venues (Other than STMU related venues and Hackathon related venues) for and running the societies official events.
  \item The Year Representatives shall be responsible for communicating to and promoting the society to students in their year.
  \item The Hack Secretary shall be responsible for organising and assembling an organisation team/SIG for any hackathons the committee wishes to run.
  \item The Technical Secretary shall be responsible for maintaining any servers and services that the society is running for both the committee and members. The Tech Sec will also be responsible for any technical setups for society events.
  \item The Graphic Designer shall be responsible for the creation of art, logos, and other designs required by the society for events, merchandise and other promotional material.
  \end{enumerate}

  It is important to note that the above are suggestions to what those elected to the positions should be responsible, not absolute rules. Delegation is encouraged and necessary, but those in the relevant positions should take responsibility for delegating the tasks and making sure they get done.

\item Non elected positions on the committee are as follows:
  \begin{enumerate}
  \item To ensure cooperation between the School of Informatics and CompSoc, the school convenors for the School of Informatics automatically have a place and vote on the committee, to vote on behalf of the school. It is entirely up to the representative to what degree they wish to participate in the committee.
  \item An EDI Representiative role created and managed by the School of Informatics will automatically have a place and vote on the committee.
  \item The leaders of the Special Interest Groups have a place and vote on the committee (see below, \enquote{Special Interest Groups}).
  \end{enumerate}

\item The committee may find it helpful to appoint members to additional positions on the committee after the elections. These must be co-opted onto the committee by a two third majority of the current committee.

\item In the event of ambiguity or contradiction within this constitution, a quorate committee may vote, by simple majority, to settle any disputes.

\end{enumerate}
