\section{Special Interest Groups (SIGs)}

\subsection{Purpose}
Special Interest Groups (SIGs) shall exist as an extension of CompSoc, functioning as mini-societies that answer to CompSoc, enabling members of CompSoc to pursue their specialised interests within computer science.

\subsection{Recognised SIGs}
The list of currrently active CompSoc SIGs shall be clearly listed on the official society website.

\subsection{Creation of SIGs}
To form a new SIG, CompSoc members must present a written proposal to the CompSoc committee detailing:
\begin{enumerate}
    \item The clear mission, scope and purpose of the new SIG's activities and how they align to CompSoc's objectives.
    \item The names of at least 10 current CompSoc members who have expressed commitment to actively participate in the SIG if approved.
    \item The name of the inaugural SIG leader, who must be a current CompSoc member. It is  encouraged that this leader be democratically selected by prospective SIG members. 
    \item The formation of a new SIG must be approved by a two-thirds majority vote of the current CompSoc committee.  
    \item In extraordinary cases only, the committee may vote to override the 10 active member minimum rule for a proposed SIG, but reasoning must be documented in meeting minutes.
\end{enumerate}

\subsection{SIG Benefits and Expectations}  
\begin{enumerate}
    \item SIGs must have leaders. SIG leaders must be EUSA members of CompSoc.
    \item Each SIG shall have full autonomy over their own member recruiting, operations, meeting cadence, and all other matters of self-governance, as long as it is with accordance with CompSoc's and EUSA's guidelines.  
    \item Approved SIGs shall be eligible to request financial support from the CompSoc budget and other non-monetary assistance from the committee. SIG leaders must provide written funding proposals to the committee including the purpose and amount requested.
    \item SIG leaders shall be expected to keep the CompSoc committee well-informed of key SIG activities, metrics and outcomes.
\end{enumerate}

\subsection{SIG Governance and Handover}
\begin{enumerate}
    \item If concerns are raised about SIG leadership that cannot be internally resolved, the CompSoc committee shall mediate a resolution process and may appoint an interim leader if needed to restore SIG function.
    \item If a SIG leader steps down or is replaced, it is their responsibility to inform the CompSoc Secretary and ensure a smooth handover to their successor, including all SIG materials and accounts.
\end{enumerate}

\subsection{SIG Conduct}
\begin{enumerate}
    \item SIGs and their members shall be subject to the same Code of Conduct and disciplinary sanctions as core CompSoc members (see 1.9).
    \item Each SIG must follow the code of conduct detailed in the \blockquote{SIG Handbook}, issued by CompSoc to every SIG leader.
\end{enumerate}

\subsection{SIG Representation in CompSoc}
\begin{enumerate}
    \item Each SIG is granted one voting seat on the CompSoc committee.
    \item The SIG seat must be filled by a EUSA CompSoc member of CompSoc.
    \item The SIG leader is expected to fill the SIG seat, attend CompSoc meetings, and vote on the SIGs behalf.
    \item If the SIG leader is unable to attend a CompSoc meeting, they shall designate a temporary representative from that SIG to the CompSoc committee to vote and be present on the SIG's behalf.
\end{enumerate}

\subsection{SIG Dissolution from CompSoc}
\begin{enumerate}
    \item If a SIG becomes dormant or fails to meet its expectations detailed in the \blockquote{SIG Handbook}, the CompSoc committee may place it under review for 1 year and require the submission of a remediation plan to avoid dissolution.
    \item The CompSoc committee shall reserve the right to dissolve a SIG by two-thirds majority vote if it remains dormant or non-compliant after a review year.
    \item If a SIG decides to voluntarily dissolve, the SIG leader must present a written letter of dissolution to the CompSoc committee. The letter of dissolution also acts as a resignation from the SIG leader. A 30 day grace period will be enacted for any SIG member to submit an objection to the committee before proceeding. If no objections are raised and the CompSoc committee cannot find a successor to the SIGs committee, the committee shall be dissolved.
    \item Upon a SIG's dissolution, any residual funding, assets or infrastructure must be promptly returned to CompSoc. Exceptions of this rule may be granted by the committee.
\end{enumerate}

\subsubsection{SIG Disassociation}
\begin{enumerate}
    \item If a SIG decides to voluntarily leave CompSoc and disassociate from its activities, the SIG leader must present a written letter requesting to leave.
    \item A 30 day grace period will be enacted for any SIG member to submit an objection to the committee before proceeding.
    \item If no objections are raised at the end of 30 days, the SIG's relationship to CompSoc will be removed, and the SIG will be considered dissolved.
\end{enumerate}

\subsection{SIG Collaboration}
\begin{enumerate}
    \item SIG leaders are encouraged to collaborate on joint events, projects and knowledge-sharing between their groups.
    \item If multiple SIGs merge, or if a SIG splits into multiple subgroups, the leaders must provide written notification to the CompSoc Secretary detailing the future leadership structure and operational implications of this change for committee approval.
\end{enumerate}

\subsection{Affiliated Groups}

The above SIG guidelines only apply internally to CompSoc's own SIGs. External groups, such as student branches of industry organisations, may seek to affiliate with CompSoc for a mutually beneficial partnership. In such cases, the CompSoc committee shall decide appropriate terms on a case-by-case basis. 