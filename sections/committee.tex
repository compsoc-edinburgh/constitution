\section{The Committee}

\begin{enumerate}

\item All office-bearers shall be subject to election annually.

\item The General Committee
  \begin{itemize}
  \item The business of the Society shall be managed by a committee of Office Bearers.
  \item Any member of the Society shall be entitled to sit on the committee.
  \item The Office Bearers must be members of the Society.
  \item All Office Bearers will complete annual online training as outlined by the Activities Team.
  \end{itemize}

\item The Executive Committee
  \begin{itemize}
  \item The President, Secretary and Treasurer of the society shall be matriculated students of The University of Edinburgh.
  \item The committee must consist of a President, Secretary, and Treasurer (the Executive Officers) as a minimum.
  \item These Office Bearers will be elected at the AGM\@.
  \item No person may be elected to more than one of these posts simultaneously.
  \item Should any of the Executive Committee resign a replacement or acting replacement will be voted in at an EGM\@.
  \end{itemize}
\end{enumerate}

\newpage

\subsection{Committee Roles}
The elected CompSoc committee shall be composed of the following roles:
\begin{enumerate}
    \item President  
    \item Vice President
    \item Secretary
    \item Treasurer
    \item Technical Secretary
    \item Social Secretary
    \item Graphic Designer
    \item Equality, Diversity \& Inclusion (EDI) Representative
    \item Social Media Officer
    \item Academic Representatives\footnote{ The year representatives must be students which will be enrolled in that year the semester following the AGM\@. At the time of the AGM the Second Year Representative should be a first year, etc.}:
    \begin{enumerate}
        \item First Year Representative\footnote{ The First Year (Freshers') Representative is a special case and will be elected at the first STMU/EGM in the first semester before the end of October. If the position is not filled the committee may select a willing first year or hold subsequent re-elections.}
        \item Second Year Representative
        \item Third Year Representative 
        \item Fourth Year Representative
        \item ``Old Person'' Representative
    \end{enumerate}
\end{enumerate}

\newpage


\subsection{Committee Duties}
All committee members shall be expected to attend weekly committee meetings and society events during the academic terms (see section 2.6).
\begin{enumerate}
    \item The President shall:
    \begin{itemize}
        \item Have overall accountability for the successful leadership of CompSoc and chair committee and General Meetings.  
        \item Serve as the key liaison with the School of Informatics and EUSA, including verifying the submission of all required documentation.
    \end{itemize}
    \item The Secretary shall:
    \begin{itemize}
        \item Manage official CompSoc correspondence and communications on behalf of the committee.
        \item Prepare agendas and keep detailed minutes for all committee meetings, AGMs and EGMs.  
    \end{itemize}
    \item The Treasurer shall:
    \begin{itemize}
        \item Oversee the CompSoc banking account and maintain detailed financial records for annual reporting and auditing.
        \item Prepare and deliver current year financial reports.
    \end{itemize}
    \item The Vice President shall:
    \begin{itemize}
        \item Support the President in overseeing all CompSoc operations, stepping in to lead meetings or events in their absence.  
        \item Manage relationships with the SIG leaders and support their activities.
        \item Coordinate speaker booking and venue logistics for tech meetups (STMUs).
    \end{itemize}
    \item The Technical Secretary shall:
    \begin{itemize}
        \item Administer CompSoc's technical infrastructure including website, servers, and digital platforms/tools.
        \item Advise on and implement appropriate technology solutions to support CompSoc operations and events.  
    \end{itemize}
    \item The Social Secretary shall:
    \begin{itemize}
        \item Plan and execute all CompSoc social events including venue booking, catering, entertainment, promotion and on-the-day logistics (excluding STMUs).
    \end{itemize}
    \item The Graphic Designer shall: 
    \begin{itemize}
        \item Create compelling visual designs for event promotion, social media, merchandise and other CompSoc media.
        \item Ensure CompSoc's brand and visual identity stays professional and consistent across all applications.
    \end{itemize}
    \item The Equality, Diversity and Inclusion (EDI) Representative shall:
    \begin{itemize}
        \item Proactively work to ensure CompSoc events and communications are welcoming and inclusive to students of all backgrounds. 
        \item Serve as a confidential point of contact for members to discuss any issues of discrimination, harassment or exclusionary behavior.
        \item Plan and implement initiatives to broaden the diversity of CompSoc membership.  
    \end{itemize}
    \item The Social Media Officer shall:  
    \begin{itemize}
        \item Manage CompSoc's social media channels and digital marketing presence to engage members and promote events/activities.
        \item Monitor social media insights and analytics to optimize CompSoc's online member engagement.
    \end{itemize}
    \item The Academic Year Representatives (First Year, Second Year, Third Year, Fourth Year, ``Old Person'') shall:
    \begin{itemize}
        \item Represent the specific interests and needs of their respective academic year-group within CompSoc.
        \item Proactively promote CompSoc membership and engage their cohort peers to boost event attendance.  
        \item Gather feedback and ideas from their cohort to inform future CompSoc offerings.
    \end{itemize}
\end{enumerate}

It is important to note that the above are suggestions to what those elected to the positions should be responsible, not absolute rules. Delegation is encouraged and necessary, but those in the relevant positions should take responsibility for delegating the tasks and making sure they get done. 

\subsection{Handover and Training}
\begin{enumerate}
    \item Following the AGM election, outgoing committee members shall be responsible for the effective handover of all records, account access and assets to their incoming counterpart.
    \item Incoming committee members shall be responsible for completing all required EUSA officer training prior to commencing their role.  
\end{enumerate}

\subsection{Non-Elected Committee Roles}
\begin{enumerate}
    \item SIG representatives (SIG leader or designated proxy) are part of the committee and may vote on any committee votes, but their presence or absence does not count towards the quorum quota.
    \item The CompSoc committee may from time-to-time invite non-elected individuals to join the committee in an advisory capacity for their specific expertise.
    \item Such non-elected roles may be added by a two-thirds majority vote of the current elected committee, and may carry any voting rights. If voting rights are granted, the vote would not count towards the quorum quota.
\end{enumerate}
\vfill

\subsection{Committee Meetings}

\subsubsection{Meeting Frequency}
The CompSoc committee shall meet at least every two weeks during the academic term to ensure the smooth running of the society.

\subsubsection{Meeting Agenda}  
\begin{enumerate}
    \item By default, committee meetings shall be open to the wider CompSoc membership to attend as observers, unless otherwise specified.
    \item The committee may vote at the start of the meeting (simple majority) to hold a closed-door session for the discussion of sensitive or confidential matters as appropriate.  
\end{enumerate}

\subsubsection{Quorum and Voting}
\begin{enumerate}
    \item Quorum for a committee meeting shall require a minimum of 75\% of elected CompSoc committee roles to be present (see 2.1 for all the roles that are counted), including at least 2 of the 3 executive officers.
    \item Committee members may designate a proxy to vote on their behalf by notifying the Secretary in writing prior to the meeting.
    \item Major permanent decisions require a 2/3 majority vote of a quorate committee. These decisions include constitution changes, creation and dissolution of SIGs, and any decisions that are deemed irreversible or highly sensitive in nature.
    \item Simple decisions do not require a vote, unless there is expressed opposition from members of CompSoc, in which case a two-thirds majority quorate committee vote is required.
    \item In the event of ambiguity or contradiction within this constitution, a quorate committee may vote, by simple majority, to settle any disputes.
\end{enumerate}

\subsubsection{Meeting Records}
\begin{enumerate}
    \item The Secretary (or a designated minute-taker in their absence) shall take comprehensive minutes at each committee meeting, which shall be circulated to the full committee for review within 48 hours.
    \item A redacted version of the minutes, removing any sensitive/confidential items, shall be posted publicly on the CompSoc website within 7 days of the meeting.
\end{enumerate}

\subsection{Penalties on Committee Members}
\subsubsection{Absence Without Leave}
\begin{enumerate}
\item If a committee member is deemed to be absent for more than 4 consecutive weeks, the committee may vote by two-thirds majority to suspend their voting rights (and remove them from the quorum minimum quota) until they resume active participation.
\item If a committee member remains absent without explanation for more than 8 consecutive weeks, the committee may vote by two-thirds majority to permanently remove them from their role and trigger a by-election at the next EGM.
\end{enumerate}

\subsubsection{Failure to Perform Duties}
\begin{enumerate}
\item If a committee member consistently fails to fulfill the responsibilities of their role as outlined in section 2.2, any of the executive officers may vote to issue them a formal written warning.
\item If the committee member's performance does not improve within 4 weeks of the written warning and provides no explanation of their performance, the committee may vote by two-thirds majority to permanently remove them from their role and trigger a by-election at the next EGM.
\end{enumerate}

\subsection{Impeachment of Committee Members}
\subsubsection{Impeachment Trigger}
A committee member may be removed from their role through an impeachment process initiated by one of the following:
\begin{enumerate}
\item A motion signed by at least 15 current CompSoc student members detailing the reasons for impeachment. This must be followed by a discussion with the representative(s) of the motion and a two-thirds majority vote of the current CompSoc committee.
\item A motion passed by a two-thirds majority of the current CompSoc committee.
\item A formal written request from the Head of Student Services at the School of Informatics detailing the reasons for impeachment.
\end{enumerate}

\subsubsection{Impeachment Process}
\begin{enumerate}
\item Upon an impeachment motion being raised, an EGM must be called within 14 days during term time (see section 5).
\item The committee member facing impeachment must be notified in writing of the motion and given at least 7 days to prepare a defense.
\item At the EGM, the impeachment motion shall be presented, along with a statement from the committee member facing impeachment if they choose to provide one.
\item Following questions and debate by the EGM attendees, a vote shall be held on the impeachment motion, requiring a two-thirds majority to pass.
\item If the impeachment motion passes, the committee member shall be immediately removed from their role and a by-election triggered to fill the vacancy.
\item If the impeachment motion fails, no further impeachment proceedings may be brought against that committee member until the following semester, unless substantial new evidence emerges.
\end{enumerate}